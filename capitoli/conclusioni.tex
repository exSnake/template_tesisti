\chapter{Conclusioni} %\label{1cap:spinta_laterale}
% [titolo ridotto se non ci dovesse stare] {titolo completo}
%


\begin{citazione}
	Questo studio si è posto l'obiettivo di creare un conversational agent a partire da quanto già presente in letteratura e semplificandone la gestione del modello.
	
	Si è focalizzato sul linguaggio DART di cui poco è presente in rete al momento della scrittura. Il dataset stesso, che è stato il lavoro più lungo e articolato, può essere riutilizzato in molti ambiti. 
	
	I risultati ottenuti sono piuttosto soddisfacenti e dal tool creato, grazie anche all'adozione di tecniche di clean architecture e del modello di sviluppo TDD si può procedere, sia a migliorare l'interazione con lo sviluppatore, magari integrandolo in un IDE, sia a migliorare le funzionalità del modello, per esempio effettuando il fine-tuning con il dataset generato verso un modello di linguaggio più avanzato come il GPT-3 o i suoi derivati.
	
	In conclusione le strade da poter intraprendere a partire da questo studio sono molte. Possiamo però affermare che il machine learning e l'intelligenza artificiale hanno imboccato una buona direzione attraverso anche l'utilizzo del Deep Learning che sicuramente porterà in futuri piuttosto brevi a risultati di grande rilevanza nella generazione del linguaggio, sia umano (chatbot, linguaggio naturale), sia sintattico (program synthesis, linguaggio di sviluppo). 
	
	Anche se le sfumature del linguaggio sono ancora troppe per riuscire ad ottenere un  alto livello di dettaglio, il rapido avanzamento nei nuovi modelli sta rendendo molto probabile che la prossima grande svolta possa essere dietro l'angolo, con lo sviluppo delle nuove tecnologie sarà possibile raggiungere tale scopo in un futuro non molto lontano.
	
	La quarta rivoluzione industriale è iniziata e l'intelligenza artificiale farà da pioniere quindi è importantissimo incentivare e sponsorizzare la ricerca e gli studi fatti in questo ambito.

\end{citazione}

\newpage
