%\selectlanguage{english}
\begin{abstract}
I recenti progressi nelle tecnologie hardware e di comunicazione, soprattutto lo sviluppo delle architetture su infrastruttura Cloud e del Machine Learning, stanno stravolgendo il modo di interagire tra macchina e uomo. 

Lo scopo di questo lavoro di tesi è quello di esplorare queste tecnologie ed apprenderne le capacità al fine di sviluppare "YUMI", un conversational agent per lo smart development che aiuta lo sviluppatore. Tale software, infatti, permetterà allo sviluppatore di accedere in maniera quasi immediata ad una vasta quantità di funzioni.

Negli ultimi anni abbiamo assistito ad un incremento dell’uso dei chatbot in vari campi, ed è proprio a partire dal concetto di “chatbot” che è stata progettata e sviluppata una loro evoluzione, orientata all’aiuto nello sviluppo,  che si integra perfettamente anche con altri tool come gli IDE (Integrates Development Environments).

L’ingegnerizzazione del software, ovvero la generazione automatica del codice sorgente, è il sogno di una vita per molti ricercatori dell'ambito informatico e del machine learning.

Oggi questa ambizione sembra diventare sempre più reale grazie a modelli di linguaggio come GPT-3. A partire dall'idea di creazione di un linguaggio di questo tipo, semplificandola, è stato pensato YUMI, un bot creato a supporto degli sviluppatori che permette di suggerire al programmatore una possibile funzione implementabile attraverso l’utilizzo di un’interfaccia.

La piattaforma su cui si basa Yumi è un’applicazione distribuita che interagisce con gli utenti mediante un’interfaccia grafica. Il sistema proposto è basato su uno stile architetturale Client-Server. Tale architettura è perfetta per lo sviluppo dell’applicazione desktop e permette in futuro la facile integrazione all'interno di un IDE poiché separa la logica di business dalla logica di presentazione.

\end{abstract} 